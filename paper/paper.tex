\documentclass[11pt,a4paper]{article}
\usepackage[utf8]{inputenc}
\usepackage[margin=1in]{geometry}
\usepackage{graphicx}
\usepackage{amsmath}
\usepackage{amssymb}
\usepackage{hyperref}
\usepackage{booktabs}
\usepackage{algorithm}
\usepackage{algpseudocode}
\usepackage{cite}

\title{Zen Engine: High-Performance Inference for Production AI}

\author{
    Zen Research Authors \\
    \textit{Zen Research DAO} \\
    \textit{Zoo Labs Inc (501(c)(3) Non-Profit)} \\
    San Francisco, California, USA \\
    \texttt{dev@hanzo.ai} \\
    \texttt{+1 (913) 777-4443}
}

\date{September 2025}

\begin{document}

\maketitle

\begin{abstract}
Zen Engine is a production-grade inference engine achieving 44K tokens/sec on consumer hardware. Built in Rust with support for multiple backends (CUDA, Metal, CPU), Zen Engine provides OpenAI-compatible APIs while supporting PyTorch, MLX, and GGUF model formats. With sub-millisecond latency and efficient memory usage, Zen Engine enables real-time AI applications on edge devices to data centers.
\end{abstract}

\section{Introduction}

Deploying AI models in production requires balancing performance, compatibility, and ease of use. Existing inference engines often sacrifice one for the other: PyTorch is flexible but slow, specialized engines are fast but inflexible. Zen Engine combines the performance of specialized engines with the compatibility and ease of use developers expect.

\subsection{Motivation}
Production AI deployments face critical challenges: (1) Inference latency affects user experience, (2) Memory usage limits deployment options, (3) API compatibility determines integration effort, (4) Format support affects model selection. Zen Engine addresses all these challenges in a single, unified engine.

\subsection{Contributions}
Our key contributions are:
\begin{itemize}
    \item 44K tokens/sec throughput on M3 Max (Apple Silicon)
    \item OpenAI-compatible REST API for drop-in replacement
    \item Support for PyTorch, MLX, and GGUF formats
\end{itemize}

\section{Related Work}

See individual model citations in bibliography.

\section{Architecture}

Zen Engine uses a layered architecture: (1) Format Layer for PyTorch/MLX/GGUF loading, (2) Backend Layer with optimized kernels for each platform, (3) Inference Layer with batching and caching, (4) API Layer with OpenAI compatibility. All layers are written in Rust for safety and performance.

\subsection{Model Design}
Detailed in Architecture section above.

\subsection{Technical Specifications}
\begin{table}[h]
\centering
\begin{tabular}{@{}ll@{}}
\toprule
\textbf{Parameter} & \textbf{Value} \\
\midrule
Throughput (M3 Max) & 44K tokens/sec \\\\\nThroughput (RTX 4090) & 28K tokens/sec \\\\\nLatency (first token) & <10ms \\\\\nFormats & PyTorch, MLX, GGUF \\\\\nBackends & CUDA, Metal, CPU \\\\\nAPI & OpenAI-compatible REST \\\\
\bottomrule
\end{tabular}
\caption{Technical specifications of engine}
\label{tab:specs}
\end{table}

\section{Training Methodology}

All training performed with Zen Gym platform.

\subsection{Training Infrastructure}
All models are trained using \textbf{Zen Gym}~\cite{zengym2025}, our unified training platform supporting:
\begin{itemize}
    \item LoRA, QLoRA, DoRA for efficient fine-tuning
    \item GRPO, GSPO for memory-efficient reinforcement learning
    \item DPO, PPO, KTO, ORPO, SimPO for alignment
    \item Unsloth for 2-5x training speedup
    \item FlashAttention-2 and Liger Kernel optimizations
\end{itemize}

\section{Experimental Results}

Zen Engine achieves 44K tokens/sec on M3 Max (MLX), 28K tokens/sec on RTX 4090 (CUDA), and 8K tokens/sec on CPU-only systems. Latency is sub-10ms for first token with proper caching. Memory usage is optimized through quantization support (Q2\_K to F16).

\subsection{Performance Benchmarks}
\begin{table}[h]
\centering
\begin{tabular}{@{}lcc@{}}
\toprule
\textbf{Benchmark} & \textbf{engine} & \textbf{Baseline} \\
\midrule
See Results section for detailed benchmarks.
\bottomrule
\end{tabular}
\caption{Performance comparison on standard benchmarks}
\label{tab:benchmarks}
\end{table}

\section{Inference and Deployment}

Models are deployed using \textbf{Zen Engine}~\cite{zenengine2025}, our high-performance inference engine achieving:
\begin{itemize}
    \item 44K tokens/sec on M3 Max (MLX backend)
    \item 28K tokens/sec on RTX 4090 (CUDA backend)
    \item OpenAI-compatible API
    \item Support for PyTorch, MLX, and GGUF formats
\end{itemize}

\section{Applications and Use Cases}

Wide range of applications across research and production.

\section{Ethical Considerations}

As a 501(c)(3) non-profit organization, Zen Research is committed to:
\begin{itemize}
    \item \textbf{Open Access}: All models released under Apache 2.0
    \item \textbf{Environmental Responsibility}: Eco-friendly training and deployment
    \item \textbf{Privacy}: Local-first inference, no data collection
    \item \textbf{Transparency}: Full disclosure of training data and methods
    \item \textbf{Safety}: Comprehensive evaluation and red-teaming
\end{itemize}

\section{Zen AI Ecosystem}

This model is part of the complete Zen AI ecosystem:

\textbf{Language Models}:
\begin{itemize}
    \item zen-nano-0.6b: Lightweight edge model
    \item zen-eco-4b-instruct: Efficient instruction-following
    \item zen-eco-4b-thinking: Chain-of-thought reasoning
    \item zen-agent-4b: Tool-calling with MCP support
\end{itemize}

\textbf{3D \& World Generation}:
\begin{itemize}
    \item zen-3d: Controllable 3D asset generation
    \item zen-voyager: Camera-controlled world exploration
    \item zen-world: Large-scale world simulation
\end{itemize}

\textbf{Video Generation}:
\begin{itemize}
    \item zen-director-5b: Text/image-to-video
    \item zen-video: Professional video synthesis
    \item zen-video-i2v: Image-to-video animation
\end{itemize}

\textbf{Audio Generation}:
\begin{itemize}
    \item zen-musician-7b: Music generation from lyrics
    \item zen-foley: Video-to-audio Foley effects
\end{itemize}

\section{Conclusion}

We presented engine, demonstrating state-of-the-art performance.

\subsection{Future Work}
Continued optimization and feature development.

\section*{Acknowledgments}

Based on open-source contributions from the community.

We thank the open-source community and our upstream contributors.

\bibliographystyle{plain}
\bibliography{paper}

\end{document}