\documentclass[11pt,a4paper]{article}
\usepackage[utf8]{inputenc}
\usepackage[margin=1in]{geometry}
\usepackage{graphicx}
\usepackage{amsmath}
\usepackage{amssymb}
\usepackage{hyperref}
\usepackage{booktabs}
\usepackage{algorithm}
\usepackage{algpseudocode}
\usepackage{cite}

\title{Zen Engine: High-Performance Inference}

\author{
    Zen Research Authors \\
    \textit{Zen Research DAO} \\
    \textit{Zoo Labs Inc (501(c)(3) Non-Profit)} \\
    San Francisco, California, USA \\
    \texttt{dev@hanzo.ai} \\
    \texttt{+1 (913) 777-4443}
}

\date{September 2025}

\begin{document}

\maketitle

\begin{abstract}
Production-grade inference achieving 44K tokens/sec with OpenAI-compatible APIs.
\end{abstract}

\section{Introduction}
Production-grade inference achieving 44K tokens/sec with OpenAI-compatible APIs.

\subsection{Key Features}
\begin{itemize}
    \item 44K tokens/sec on M3 Max (Apple Silicon)
    \item OpenAI-compatible REST API
    \item PyTorch, MLX, and GGUF format support
    \item Multi-backend: CUDA, Metal, CPU
\end{itemize}

\section{Technical Specifications}
\begin{table}[h]
\centering
\begin{tabular}{@{}ll@{}}
\toprule
\textbf{Parameter} & \textbf{Value} \\
\midrule
Throughput (M3 Max) & 44K tokens/sec \\
Throughput (RTX 4090) & 28K tokens/sec \\
Latency (first token) & <10ms \\
Formats & PyTorch, MLX, GGUF \\
Backends & CUDA, Metal, CPU \\
API & OpenAI-compatible REST
\bottomrule
\end{tabular}
\caption{Technical specifications}
\label{tab:specs}
\end{table}

\section{Zen AI Ecosystem}

This is part of the complete Zen AI hypermodal ecosystem:

\textbf{Language Models}: zen-nano-0.6b, zen-eco-4b-instruct, zen-eco-4b-thinking, zen-agent-4b

\textbf{3D \& World}: zen-3d, zen-voyager, zen-world

\textbf{Video}: zen-director-5b, zen-video, zen-video-i2v

\textbf{Audio}: zen-musician-7b, zen-foley

\textbf{Infrastructure}: Zen Gym (training), Zen Engine (inference)

\section{Conclusion}
Zen Engine delivers production-grade inference with 44K tokens/sec throughput and sub-10ms latency.

\section*{Acknowledgments}
We thank the open-source community and our upstream contributors.

\bibliographystyle{plain}
\bibliography{paper}

\end{document}
